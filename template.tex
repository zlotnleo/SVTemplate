\documentclass[10pt,twoside,a4paper]{article}

% Configure these parameters.
\newcommand{\studentname}{Harry~Potter}
\newcommand{\studentemail}{hjp1@cam.ac.uk}
\newcommand{\studentcollege}{Pelby College}
\newcommand{\svtripos}{Computer Science Tripos}
\newcommand{\svtripospart}{IA}
\newcommand{\svcourse}{Dark Arts}
\newcommand{\svcoursefull}{Defence Agains the Dark Arts}
\newcommand{\svnumber}{1}
\newcommand{\svdate}{15.11.2019}
\newcommand{\svdatefull}{15 November 2019}
\newcommand{\svtime}{14:00}
\newcommand{\svvenuefull}{Porterhouse College, 42M}
\newcommand{\svvenue}{Porterhouse 42M}
\newcommand{\svrname}{Dr~Sup~E.~R.~Visor}
% End configuration

\usepackage{authblk}
\usepackage{titling}

\setlength{\droptitle}{-6em}
\title{\large\textbf{\svtripos\ Part~\svtripospart\\\svcoursefull\\Supervision~\svnumber}}
\date{\textbf{\svdatefull\ \svtime\\\svvenuefull}}
\author{\textbf{\studentname} \\ \studentemail}
\affil{\studentcollege\\University~of~Cambridge}

%\usepackage{a4}             % Adjust margins for A4 media
\usepackage{fancyhdr}
\renewcommand{\headrulewidth}{0.4pt}
\renewcommand{\footrulewidth}{0.4pt}
\fancyheadoffset[LO,LE,RO,RE]{0pt}
\fancyfootoffset[LO,LE,RO,RE]{0pt}
\pagestyle{fancy}
\fancyhead{}
\fancyhead[R]{\studentname\\\studentemail}
\fancyhead[L]{\svtripospart\ \svcourse, SV~\svnumber\\\svdate\ \svtime, \svvenue}
\fancyfoot{}
\fancyfoot[R]{\thepage\ / \pageref{LastPage}}

%Remove the header on the first page
\fancypagestyle{plain}{
    \fancyhf{} % clear all header and footer fields
    \renewcommand{\headrulewidth}{0pt}
    \fancyhead[R]{For \svrname}
    \fancyfoot[R]{\thepage\ / \pageref{LastPage}}
}

\usepackage{lastpage}       % "n of m" page numbering
\usepackage{lscape}         % Makes landscape easier
%\usepackage{portland}      % Switch between portrait and landscape
\usepackage{graphics}       % Graphics commands
\usepackage{wrapfig}        % Wrapping text around figures
\usepackage{epsfig}         % Embed encapsulated postscript
\usepackage{rotating}       % Extra graphics rotation
%\usepackage{tables}        % Tabular environments
\usepackage{longtable}      % Page breaks within tables
\usepackage{supertabular}   % Page breaks within tables
\usepackage{multicol}       % Allows table cells to span cols
\usepackage{multirow}       % Allows table cells to span rows
\usepackage{texnames}       % Macros for common tex names
%\usepackage{trees}         % Tree-like layout
\usepackage{hhline}         % Horizontal lines in tables
\usepackage{siunitx}        % Correct spacing of units

\usepackage{listings}       % Source code listings
\usepackage{array}          % Array environment
\usepackage{hyperref}       % URL formatting
\usepackage{amsmath}        % American Mathematical Society
\usepackage{amssymb}        % Maths symbols
\usepackage{amsthm}         % Theorems
%\usepackage{mathpartir}    % Proofs and inference rules
\usepackage{verbatim}       % Verbatim blocks
\usepackage{ifthen}         % Conditional processing in tex
\usepackage{xcolor}         % X11 colour names

\usepackage{enumitem}
\usepackage{mdframed}
\usepackage{forest}
\usepackage{tabularx}
\usepackage{calc}
\usepackage{arydshln}
\usepackage{cancel}
\usepackage{xifthen}
\usepackage{caption}
\usepackage{float}
\usepackage[section]{placeins} % control floats placement
\usepackage{minted}
\usepackage{wasysym}
\usepackage{adjustbox}

\usepackage[T1]{fontenc}

\usepackage{tikz}
\usepackage{tikz-dependency}
\usetikzlibrary{arrows,automata}


% control width and vertically align text in table cells
\newcolumntype{L}[1]{>{\raggedright\let\newline\\\arraybackslash\hspace{0pt}}m{#1}}
\newcolumntype{C}[1]{>{\centering\let\newline\\\arraybackslash\hspace{0pt}}m{#1}}
\newcolumntype{R}[1]{>{\raggedleft\let\newline\\\arraybackslash\hspace{0pt}}m{#1}}

% make hyperref links not-ugly
\hypersetup{
    colorlinks=false,
    pdfborder={0 0 0},
}

\setlength{\oddsidemargin}{-20pt}
\setlength{\evensidemargin}{-20pt}
\setlength{\topmargin}{-30pt}
\addtolength{\textheight}{100pt}
\setlength{\textwidth}{500pt}

\setlength{\parindent}{0em}
\addtolength{\parskip}{1ex}

% Define questions environment with framed \question
\newcommand{\questionlabel}{Question }
\newcommand{\restorequestionlabel}{\renewcommand{\questionlabel}{Question }}
\newcommand{\setquestionlabel}[1]{\renewcommand{\questionlabel}{#1}}
\newlist{questions}{enumerate}{4}
\setlist[questions]{align=left,ref=\questionlabel\arabic*,label=\textbf{\questionlabel\arabic*.},wide}
\newcommand{\itembr}{\needspace{12\baselineskip}\item\mbox{}}
\newcommand{\questionmarks}[1]{\hspace*{\fill}[#1~mark\ifthenelse{\equal{#1}{1}}{}{s}]}
\newcommand{\question}[2][]{\itembr\begin{mdframed}#2
\ifthenelse{\equal{#1}{}}{}{\questionmarks{#1}}
\end{mdframed}}

% Same as \question but in environment form
% e.g. for use with verbatim environments inside
\newenvironment{questionenv}[1][]{
    % save macro since \endquestionenv cannot access arguments
    \def\questionenvmarks{\ifthenelse{\equal{#1}{}}{}{\questionmarks{#1}}}
    \itembr
    \begin{mdframed}
}{
    \questionenvmarks
    \end{mdframed}
}

\renewcommand{\thefootnote}{\fnsymbol{footnote}}

\newcommand*{\bfrac}[2]{\genfrac{}{}{0pt}{0}{#1}{#2}} % Fractions without a bar

\newcommand{\Wat}{$\cal W$\kern-.5em\lower.7ex\hbox{$\cal A$}\kern-.25em$\cal T$\kern-.35em\lower.7ex\hbox{\textbf{?}}\kern-0.2em\textbf{?}}
\newcommand{\Wtf}{\textbf{W}\kern-.25em\raise.7ex\hbox{\textbf{T}}\kern-.25em\textbf{F}\kern-.3em\lower1ex\hbox{\textbf{?}}\kern-0.2em\textbf{?}}

%\usetikzlibrary{external}
%\tikzexternalize[prefix=tikz/]

\begin{document}
\maketitle
\rule{\textwidth}{1pt}

%%%%%%%%%%%%%%%%%%%%%
%
% Set configuration settings at top of file.

\section{Section One}
\begin{questions}[label=(\textit{\alph*})]
\question[3]{Draw a truth table for logical XOR?}
\begin{center}
\begin{tabular}{| c | c | c |}
\hline
$p$ & $q$ & $p\land q$ \\
\hline
0 & 0 & 0 \\
0 & 1 & 1 \\
1 & 0 & 1 \\
1 & 1 & 0 \\
\hline
\end{tabular}
\end{center}

%Use environment instead of \question because it has code inside
\begin{questionenv}[10]
Consider the following ML functions:

\begin{minted}{sml}
fun p f s (t::h) = (f h) t s @ p (f p (r f))::q h
and q a b = (b a p (p q) a) b :: (q b);
\end{minted}

\begin{enumerate}[label=(\textit{\roman*})]
\item What is the ML type associated with p?
\item What does the function \texttt{p} do?
\item Why does it do it?
\item How long does it take?
\end{enumerate}
\end{questionenv}
\begin{enumerate}[label=(\textit{\roman*})]
\item $\ldots$
\item $\ldots$
\item $\ldots$
\item $\ldots$
\end{enumerate}
\end{questions}

\newpage
\section{Section Two}

\begin{questions}[label=]
\question[10]{Do some maths}

$$\int\sin(x) dx =-\cos(x)+C$$

$$\int x^2 dx =\frac{x^3}{3}+C$$

\question[10]{Give an example of a 5-colourable graph and write some words}

\begin{minipage}[t]{0.5\linewidth}
\phantom{invisible text}
\begin{center}
\begin{tikzpicture}[node distance=1.75cm,every node/.style={draw,circle,minimum size=0.8cm,inner sep=0pt}]
\node[fill=red] (y) {$y$};
\node[fill=blue,text=white,below of=y] (x) {$x$};
\node[fill=green,left of=x,xshift=0.25cm] (t5) {$t_5$};
\node[fill=red,below of=x] (c) {$c$};
\node[fill=red,left of=c] (t4) {$t_4$};
\node[fill=magenta,right of=c] (t3) {$t_3$};
\node[fill=green,below of=t4] (a) {$a$};
\node[fill=yellow,below of=t3] (b) {$b$};
\node[fill=blue,text=white,below right of=a,xshift=-0.25cm] (t1) {$t_1$};
\node[fill=blue,text=white,below left of=b,xshift=0.25cm] (t2) {$t_2$};
\draw
    (a) -- (b) -- (c) -- (a)
    (a) -- (t1) -- (b)    (c) -- (t1)
    (a) -- (t2) -- (b)    (c) -- (t2)
    (a) -- (x) -- (b)    (c) -- (x)
    (a) -- (t3) -- (b)    (c) -- (t3) -- (x)
    (a) -- (t4) -- (b)    (x) -- (t4)
    (x) -- (y) edge[out=180,in=120] (a)    (y) edge[out=0,in=60] (b)
    (x) -- (t5) -- (y)
;
\end{tikzpicture}
\end{center}
\end{minipage}\begin{minipage}[t]{0.5\linewidth}
\phantom{invisible text}
{\Large ``some words''}
\begin{itemize}
\item words
\item words
\item more words
\end{itemize}
\end{minipage}

\end{questions}

% End of your supervision work
%
%%%%%%%%%%%%%%%%%%%%%

\end{document}

